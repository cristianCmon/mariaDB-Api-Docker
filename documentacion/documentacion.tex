% [Tamaño principal de la fuente del documento, tamaño del papel, con título (crea salto de página)]{Tipo de documento}
\documentclass[12pt, a4paper, titlepage]{article}
\usepackage[utf8]{inputenc}
% Traduce expresiones al español
\usepackage[spanish]{babel}
\usepackage{setspace}

\usepackage{graphicx}
\usepackage{geometry}
\geometry{margin=2.5cm}
\graphicspath{ {./capturas/} }

% Permite utilizar labeling para listar de forma personalizada
\usepackage{scrextend}
% Permite centrar verticalmente m{}
\usepackage{array}

\title{\LARGE \textbf{API Online} \\[2ex] \Large Afondamento nas Competencias Profesionais}
\author{\\[20ex]Cristian Fernández}
\date{23 de enero de 2026}

\begin{document}
\maketitle

\doublespacing
\tableofcontents % Índice
\newpage

\section{Introducción}
\noindent \\El objetivo de esta tarea es detallar el proceso de configuración y despliegue automatizado de un proyecto 
(contenedor \textit{Docker} \textit{API Express.js} + \textit{MariaDB}) alojado en un repositorio remoto (\textit{GitHub}), garantizando
que cualquier cambio en el código fuente se refleje de manera eficiente en el entorno de ejecución.\\
\noindent \\También buscaremos la \underline{Integración Continua} mediante la conexión directa entre el repositorio remoto 
y \textit{Coolify}. Este enfoque permitirá al servidor detectar automáticamente nuevas actualizaciones en la rama principal,
iniciar la construcción de las imágenes de \textit{Docker} y desplegar los contenedores sin intervención manual, mejorando sustancialmente 
la calidad de vida del trabajo consiguiendo, entre otra cosas, reducir errores humanos y optimizar los tiempos de entrega. \\
\noindent \\Para ello seguiremos los siguiente pasos:\\
\begin{itemize}
    \item Crearemos una \textbf{máquina virtual Ubuntu} (versión 24.04.2 LTS) que actuará como nodo principal de despliegue.
    \item Instalaremos y usaremos \textbf{Ngrok} para crear túneles seguros desde nuestra máquina virtual hacia internet, permitiendo 
    exponer servidores web (HTTP/HTTPS), sitios en desarrollo, webhooks y servicios TCP/SSH a una \underline{URL pública temporal}.
    \item Instalaremos y usaremos \textbf{Coolify} para gestionar el ciclo de vida de la aplicación, desde la lectura del repositorio
    hasta el monitoreo de los contenedores.
\end{itemize}
\newpage

\section{Herramientas empleadas}
\subsection{VirtualBox}
\noindent \\\textit{VirtualBox} es un hipervisor de tipo 2 gratuito y de código abierto (GPLv2), ideal para virtualización en equipos 
de escritorio. Sus características clave incluyen la capacidad de ejecutar múltiples sistemas operativos invitados 
(Windows, Linux, macOS, etc.) de forma simultánea, alta portabilidad entre plataformas, creación de "snapshots"
(instantáneas) para revertir fallos, soporte para dispositivos USB y redes virtuales.\\
\noindent \\Las características principales de \textit{VirtualBox} son las siguienes:\\
\begin{itemize}
    \item \textbf{Multiplataforma y Soporte de SOs}: Puede instalarse en Windows, macOS, Linux y Solaris (anfitrión), y virtualizar
    casi cualquier otro sistema operativo (invitado) de 32 o 64 bits.
    \item \textbf{Guest Additions}: Conjunto de controladores y aplicaciones que optimizan el rendimiento, mejoran la comunicación
    (carpetas compartidas, portapapeles compartido) y ajustan la resolución de pantalla.
    \item \textbf{Amplias Opciones de Red}: Soporta modos NAT, puente (bridged), red interna, y red exclusiva del anfitrión (host-only)
    para adaptar la conectividad de la VM.
    \item \textbf{Soporte de Hardware Virtual}: Emula hardware moderno, incluyendo soporte para USB 2.0/3.0, hasta 32 CPUs virtuales,
    y aceleración 3D.
\end{itemize}
\newpage

\subsection{Ngrok}
\noindent \\\textit{Ngrok} es una herramienta de desarrollo que crea túneles seguros desde una red pública a servidores locales,
permitiendo exponer proyectos localhost mediante URLs HTTPS sin configurar firewalls o redes. Ofrece inspección de tráfico en
tiempo real, soporte para múltiples protocolos (HTTP, TCP), y es ideal para pruebas rápidas, webhooks y demostraciones a clientes. \\
\noindent \\Las características principales de \textit{Ngrok} son las siguientes:\\
\begin{itemize}
    \item \textbf{Exposición de Localhost}: Permite que servicios locales sean accesibles desde internet a través de una URL pública.
    \item \textbf{Inspección de Tráfico}: Proporciona una interfaz web para analizar y depurar solicitudes HTTP/HTTPS entrantes.
    \item \textbf{Túneles Seguros y Rápidos}: Crea conexiones seguras y cifradas (HTTPS/TLS) hacia muestra máquina local de forma instantánea.
    \item \textbf{Soporte de Webhooks}: Facilita el desarrollo al permitir recibir notificaciones de servicios externos directamente.
    en el entorno local.
\end{itemize}
\newpage

\subsection{Coolify}
\noindent \\\textit{Coolify} es una plataforma de autoalojamiento (self-hosting) de código abierto y sin límites, diseñada
como una alternativa gratuita a Vercel o Heroku. Permite desplegar y gestionar aplicaciones web, bases de datos (PostgreSQL, MySQL,
Redis, MongoDB) y servicios en cualquier servidor (VPS, Raspberry Pi) mediante Docker. Se caracteriza por la automatización Git-Ops,
SSL gratuito, backups y una interfaz intuitiva.\\
\noindent \\Las características principales de \textit{Coolify} son las siguientes:\\
\begin{itemize}
    \item \textbf{Soporte Multi-tecnología}: Compatible con cualquier aplicación dockerizable, incluyendo Node.js, Python, PHP, Ruby on Rails, Rust y sitios estáticos.
    \item \textbf{Gestión Total de Bases de Datos}: Permite crear, gestionar y hacer copias de seguridad de bases de datos como PostgreSQL, MySQL, Redis y MongoDB, así como herramientas como WordPress o N8N.
    \item \textbf{Despliegue Automatizado (Git-Ops)}: Conexión con GitHub, GitLab o Bitbucket para realizar despliegues automáticos cada vez que se actualiza el código.
    \item \textbf{Interfaz de Usuario Sencilla}: Panel de control moderno y fácil de usar, ideal tanto para principiantes como para desarrolladores experimentados.
\end{itemize}
\newpage

\section{Instalaciones}
\noindent \\En la máquina virtual, abrimos un terminal y pegamos los códigos que se muestran en las siguientes capturas (\underline{es necesario crearse una
cuenta en cada aplicación}):\\
\begin{figure}[hbt!]
    \centering
    \includegraphics[width=1\textwidth]{Screenshot_0.png}
    % \caption{Pegamos el código subrayado en el terminal y pulsamos Enter}
\end{figure}
\begin{figure}[hbt!]
    \centering
    \includegraphics[width=1\textwidth]{Screenshot_1.png}
\end{figure}
\newpage

\section{Configuraciones y despliegues}
\subsection{Ngrok local}
\noindent \\En las siguientes capturas se muestra cómo levantar ngrok desde la terminal:\\
\begin{figure}[hbt!]
    \centering
    \includegraphics[width=1\textwidth]{Screenshot_2a.png}
    % \caption{Pegamos el código subrayado en el terminal y pulsamos Enter}
\end{figure}
\noindent \\
\begin{figure}[hbt!]
    \centering
    \includegraphics[width=1\textwidth]{Screenshot_2b.png}
    % \caption{Pegamos el código subrayado en el terminal y pulsamos Enter}
\end{figure}
\newpage

\subsection{Coolify (repositorio y base de datos)}
\noindent \\Una vez levantado Ngrok en la máquina virtual podremos acceder a Coolify en la máquina real abriendo un navegador y entrando en 
la url. Miraremos lo siguiente:\\
\begin{figure}[hbt!]
    \centering
    \includegraphics[width=1\textwidth]{Screenshot_3a.png}
    \caption{Página principal de Coolify}
\end{figure}
\begin{figure}[hbt!]
    \centering
    \includegraphics[width=1\textwidth]{Screenshot_3b.png}
    \caption{Crearemos un nuevo proyecto}
\end{figure}
\newpage

\indent \\
\begin{figure}[hbt!]
    \centering
    \includegraphics[width=1\textwidth]{Screenshot_3c.png}
    \caption{Nos aparecerá el siguiente modal e introduciremos un nombre}
\end{figure}
\indent \\
\begin{figure}[hbt!]
    \centering
    \includegraphics[width=1\textwidth]{Screenshot_3d.png}
    \caption{Pulsamos el botón de añadir recurso}
\end{figure}
\newpage

\indent \\
\begin{figure}[hbt!]
    \centering
    \includegraphics[width=1\textwidth]{Screenshot_3e.png}
    \caption{Deberemos crear 2 recursos: Uno que apunte al repositorio de nuestra API y una base de datos compatible con la de nuestra API.
    En este caso empezaremos por crear el recurso del repositorio público, aunque el orden es indiferente}
\end{figure}
\newpage

\begin{figure}[hbt!]
    \centering
    \includegraphics[width=1\textwidth]{Screenshot_4.png}
    \caption{Configuramos siguiendo las indicaciones de esta captura}
\end{figure}
\indent \\
\begin{figure}[hbt!]
    \centering
    \includegraphics[width=1\textwidth]{Screenshot_4b.png}
    \caption{Paralelamente necesitamos conocer la ip de la máquina virtual}
\end{figure}
\newpage

\indent \\
\begin{figure}[hbt!]
    \centering
    \includegraphics[width=1\textwidth]{Screenshot_4a.png}
    \caption{Seguimos las indicaciones de la captura. Posteriormente volveremos a este punto a configurar las variables de entorno}
\end{figure}
\newpage

\begin{figure}[hbt!]
    \centering
    \includegraphics[width=1\textwidth]{Screenshot_4d.png}
    \caption{Creamos el recurso de base de datos y la configuramos según la captura}
\end{figure}
\indent \\
\begin{figure}[hbt!]
    \centering
    \includegraphics[width=1\textwidth]{Screenshot_4e.png}
    \caption{En el mismo recurso configuramos la siguiente variable}
\end{figure}
\newpage

\noindent \\
\begin{figure}[hbt!]
    \centering
    \includegraphics[width=1\textwidth]{Screenshot_4c.png}
    \caption{Volvemos al recurso repositorio y añadimos las variables de entorno de la base de datos tal como se muestra en la captura}
\end{figure}
\newpage

\noindent \\Ahora sólo falta desplegar ambos recursos y comprobar si se pueden hacer peticiones a la API. Para ello introduciremos la url
de la Figura 8 (en la que insertamos la ip de la máquina virtual). El resultado es el siguiente:
\noindent \\
\begin{figure}[hbt!]
    \centering
    \includegraphics[width=1\textwidth]{Screenshot_4f.png}
    \caption{Si todo ha ido bien deberíamos visualizar correctamente la llamada a la API}
\end{figure}
\newpage

\subsection{Ngrok expuesto en Internet}
\noindent \\
\begin{figure}[hbt!]
    \centering
    \includegraphics[width=1\textwidth]{Screenshot_10b.png}
    \caption{Desde Coolify copiamos la url del dominio enlazado al repositorio}
\end{figure}

\noindent \\
\begin{figure}[hbt!]
    \centering
    \includegraphics[width=1\textwidth]{Screenshot_10.png}
    \caption{Desde un terminal levantamos Ngrok en el puerto 80 con la url anterior pasada al parámetro \textit{--host-header}}
\end{figure}
\newpage

\noindent \\
\begin{figure}[hbt!]
    \centering
    \includegraphics[width=1\textwidth]{Screenshot_11.png}
    \caption{Entramos a la url proporcionada por Ngrok desde el teléfono móvil}
\end{figure}

\noindent \\
\begin{figure}[hbt!]
    \centering
    \includegraphics[width=1\textwidth]{Screenshot_9.png}
    \caption{Conexiones satisfactorias en captura actual y anterior}
\end{figure}
\newpage

\subsection{Coolify (Webhook)}
\noindent \\Los Webhooks nos permiten, entre otras cosas, desplegar automáticamente una nueva versión de la aplicación cada vez que enviamos 
cambios (push) a una rama específica de nuestro repositorio remoto.\\
\noindent \\La configuración del lado de Coolify quedaría así:\\
\begin{figure}[hbt!]
    \centering
    \includegraphics[width=1\textwidth]{Screenshot_12.png}
    \caption{Necesitaremos estos datos para enlazarlos con GitHub}
\end{figure}
\newpage
\noindent La configuración del lado de nuestro proyecto de GitHub quedaría así:
\begin{figure}[hbt!]
    \centering
    \includegraphics[width=1\textwidth]{Screenshot_13.png}
    \caption{Importante seleccionar el formato de datos json}
\end{figure}
\noindent \\
\begin{figure}[hbt!]
    \centering
    \includegraphics[width=1\textwidth]{Screenshot_14.png}
    \caption{Una vez añadido el Webhook se hará un ping de comprobación}
\end{figure}
\newpage

\noindent \\Ahora comprobaremos el correcto funcionamiento del Webhook recién creado realizando un push a la rama principal de
nuestro repositorio:\\
\begin{figure}[hbt!]
    \centering
    \includegraphics[width=1\textwidth]{Screenshot_15.png}
    \caption{El endpoint raíz de nuestra API...}
\end{figure}
\noindent \\
\begin{figure}[hbt!]
    \centering
    \includegraphics[width=1\textwidth]{Screenshot_16.png}
    \caption{...muestra lo siguiente}
\end{figure}
\noindent \\
\begin{figure}[hbt!]
    \centering
    \includegraphics[width=1\textwidth]{Screenshot_17.png}
    \caption{Si lo modificamos...}
\end{figure}
\newpage

\noindent \\
\begin{figure}[hbt!]
    \centering
    \includegraphics[width=1\textwidth]{Screenshot_18.png}
    \caption{...y pusheamos el cambio...}
\end{figure}
\noindent \\
\begin{figure}[hbt!]
    \centering
    \includegraphics[width=1\textwidth]{Screenshot_19.png}
    \caption{...observaremos que el cambio se ha realizado automáticamente}
\end{figure}
\newpage

\end{document}