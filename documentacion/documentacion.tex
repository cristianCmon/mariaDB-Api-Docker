% [Tamaño principal de la fuente del documento, tamaño del papel, con título (crea salto de página)]{Tipo de documento}
\documentclass[12pt, a4paper, titlepage]{article}
\usepackage[utf8]{inputenc}
% Traduce expresiones al español
\usepackage[spanish]{babel}
\usepackage{setspace}

\usepackage{graphicx}
\usepackage{geometry}
\geometry{margin=2.5cm}
\graphicspath{ {./capturas/} }

% Permite utilizar labeling para listar de forma personalizada
\usepackage{scrextend}
% Permite centrar verticalmente m{}
\usepackage{array}

\title{\LARGE \textbf{Chat Grupal} \\[2ex] \Large Programación de Servicios y Procesos}
\author{\\[20ex]Cristian Fernández}
\date{21 de enero de 2026}

\begin{document}
\maketitle

\doublespacing
\tableofcontents % Índice
\newpage

\section{Descripción del problema}
\noindent \\El docente encargado de impartir el módulo de \textit{Programación de Servicios y Procesos}, Don Roberto Castro Liste,
ha encargado la realización de una tarea que consiste en crear un aplicación en \textbf{JavaFX} que permita chatear entre usuarios,
demostrando la comprensión y el dominio en el uso de sockets y también en el uso de hilos con la utilización de métodos como \textit{newFixedThreadPool()} 
de la clase \textit{ExecutorService}. \\ \\
El programa debe seguir estas pautas:\\
\begin{itemize}
    \item Crear una clase servidor que esté escuchando constantemente conexiones entrantes.
    \item Crear una clase cliente que permita enviar y recibir mensajes del servidor.
    \item Crear una clase manejadora que comunique correctamente las 2 clases anteriores.
    \item Diseñar una interfaz que permita ver de forma clara el funcionamiento del chat.
\end{itemize}
\newpage

\section{Requisitos funcionales}
\noindent \\A continuación se listan los requisitos funcionales del sistema:\\
%\begin{description} sería equivalente pero items en negrita por defecto
\begin{labeling}{ RF-5:} % el segundo parámetro marca el espaciado entre ítem y texto
    \item[ \textbf{ RF-1}:] Solicitar al usuario que introduzca su nombre antes de entrar en el chat.
    \item[ \textbf{ RF-2}:] Pulsar un botón que inicie la ejecución del programa en una nueva ventana.
    \item[ \textbf{ RF-3}:] Permitir la inserción de mensajes de forma ágil únicamente usando el teclado.
    \item[ \textbf{ RF-4}:] Mostrar correctamente y en tiempo real todos los mensajes de todos los participantes en el chat.
    \item[ \textbf{ RF-5}:] Terminar los hilos en ejecución cuando se cierre la aplicación Cliente.
\end{labeling}
\newpage

\section{Requisitos no funcionales}
\noindent \\A continuación se listan los requisitos NO funcionales del sistema:\\
\begin{labeling}{ RNF-5:}
    \item[ \textbf{RNF-1}:] El tiempo de respuesta al pulsar el botón \textit{Entrar Chat} no debe superar los 2 segundos.
    \item[ \textbf{RNF-2}:] La intefaz gráfica de usuario ha de ser fácilmente entendible.
    \item[ \textbf{RNF-3}:] El tiempo de respuesta al mandar un mensaje pulsando Enter no debe superar el segundo.
    \item[ \textbf{RNF-4}:] El usuario debe distinguir fácilmente sus mensajes del resto de mensajes de los demás usuarios.
\end{labeling}
\newpage

\section{Casos de uso}
\noindent \\A continuación se muestran los casos de uso mediante la siguiente tabla:\\ \\ \\
\begin{tabular}{| m{5cm} | m{8cm} |} % | pinta las columnas verticales
\hline % pinta las filas horizontales
\textbf{Caso de uso} & \textbf{Descripción} \\
\hline
Introducir nombre & El usuario debe introducir su nombre antes de entrar en el chat. \\
\hline
Iniciar aplicación & El usuario puede iniciar la ventana de chat pulsando un botón. \\
\hline
Escribir mensaje & El usuario puede enviar un mensaje escribiendo y pulsando Enter. \\
\hline
Cerrar aplicación & El usuario puede cerrar la aplicación por completo pulsando el botón de cerrar ventana. \\
\hline
\end{tabular}
\newpage

\section{Historias de usuario}
\noindent \\A continuación se muestran las historias de usuario mediante la siguiente tabla:\\ \\ \\
\begin{tabular}{| m{2cm} | m{2cm} | m{4cm} | m{4cm} |}
\hline
\textbf{ID} & \textbf{Como...} & \textbf{Quiero...} & \textbf{Para...} \\
\hline
HU-01 & Usuario & Insertar mi nombre & Que me identifiquen en el chat \\
\hline
HU-02 & Usuario & Pulsar un botón & Iniciar la aplicación \\
\hline
HU-03 & Usuario & Manejar el chat únicamente con el teclado & Disfrutar de una buena usabilidad \\
\hline
HU-04 & Usuario & Ver los nombres de los participantes & Saber en todo momento con quién hablo \\
\hline
\end{tabular}
\newpage

\section{Herramientas de desarrollo}
\noindent \\\textbf{IntelliJ Community Edition} es un IDE gratuito y de código abierto, ideal para el desarrollo en lenguajes JVM y Android.
Sus características principales incluyen un potente editor de código con autocompletado inteligente y refactorización, soporte para control de versiones,
un depurador integrado, y herramientas de pruebas unitarias. Está disponible para todas las plataformas.  \\
\\Características principales: \\
\begin{itemize}
    \item \textbf{Editor de código potente:} Ofrece resaltado de sintaxis, análisis en tiempo real, sugerencias de código, e inspecciones y correcciones rápidas.
    \item \textbf{Soporte para múltiples lenguajes:} Es compatible con Java, Kotlin, Groovy, Scala, y otros lenguajes JVM.
    \item \textbf{Integración con control de versiones:} Permite la integración con sistemas como Git, SVN, y Mercurial.
\end{itemize}
\newpage

\section{Lenguajes de programación}
\noindent \\\textbf{JavaFX} se caracteriza por su capacidad de crear aplicaciones enriquecidas visualmente, integrando multimedia (audio, video) y gráficos vectoriales.
Permite desarrollar aplicaciones multiplataforma (escritorio, web, móvil, smart TV). Otras características clave son
la integración fluida con el ecosistema Java y herramientas como FXML y Scene Builder para un desarrollo más eficiente y colaborativo. \\
\\Características principales: \\
\begin{itemize}
    \item \textbf{Interfaces gráficas de usuario (GUI) enriquecidas:} Permite crear interfaces modernas y atractivas con animaciones, gráficos vectoriales, y elementos personalizables.
    \item \textbf{Integración multimedia:} Facilita la inclusión de contenido de audio, video y otros medios dentro de las aplicaciones.
    \item \textbf{Desarrollo declarativo (FXML):} Utiliza FXML, un lenguaje basado en XML, para describir la interfaz de usuario, lo que facilita el trabajo colaborativo entre diseñadores y desarrolladores.
\end{itemize}
\newpage

\section{Interfaz de la aplicación}
\noindent \\A continuación unas capturas de las diferentes vistas de la interfaz.\\ \\
% \begin{figure}[hbt!]
%     \centering
%     \includegraphics[width=0.5\textwidth]{01.png}
%     \caption{Vista inicial, requiere introducir nombre usuario para continuar}
% \end{figure}
% \noindent \\ \\
% \begin{figure}[hbt!]
%     \centering
%     \includegraphics[width=0.5\textwidth]{02.png}
%     \caption{Si el campo está vacío se muestra un placeholder en el interior del TextField}
% \end{figure}
% \newpage

% \noindent \\
% \begin{figure}[hbt!]
%     \centering
%     \includegraphics[width=1\textwidth]{03.png}
%     \caption{Los textos en color son los del usuario, en cursiva los de los demás}
% \end{figure}

\end{document}